\newpage
\thispagestyle{empty}
\cleardoublepage
\chapter*{Vorwort}
\addcontentsline{toc}{chapter}{Vorwort}

\begingroup
\leftskip=10mm
\textit{�Man versteht etwas nicht wirklich,\\
wenn man nicht versucht, es zu implementieren.�}\\[2mm]
--- von \textsc{Donald Ervin Knuth} \cite{Knuth}
\vspace{5mm}
\par
\endgroup

Sehr geehrter Leser,

das obige Zitat von Donald Knuth stellt eine der Grunderfahrung dar, welche ich bei der Erarbeitung dieses Dokuments ein weiteres Mal erleben durfte. Es behandelt den grundlegenden Unterschied, ob man �ber ein Thema plappert oder sich die Strenge auferlegt, das Verst�ndnis in Quellcode "`zu gie�en"'. Denn der Computer ist einer der unbarmherzigsten Zeitgenosse, welcher jeden Fehler, jede Unsicherheit bez�glich eines ihm bekannten Themas sofort enttarnt. So sah ich mich 2 Tag vor der Abgabe dieses Dokuments mit einem einfachen SEGMENTATION FAULT konfrontiert, der mich glatt an den Rand der Verzweiflung brachte, nur um 6 Stunden sp�ter einen einfachen Tippfehler zu beheben ...

Aber zum Gl�ck empfindet der Mensch eben nicht nur die Last der Exaktheit, sondern auch deren Erf�llung, welche sich rasch nach einem Erfolg breit macht. Kann man diesen dar�ber hinaus auch noch mit seinem 2. Lieblingsfach verbinden, kann man tiefes Gl�ck erfahren.

Nach dieser kleinen Achterbahn-der-Gef�hle-Schilderung, kann ich Ihnen nur noch die selbe Freude beim Leser dieser Arbeit w�nschen, wie ich sie (abschnittsweise) beim Schreiben innehatte!

Au�erdem m�chte ich mich bei meinen gro�en Unterst�tzern, meiner Freundin und meinem Bruder und meinen beiden Eltern vom Herzen f�r jegliche Unterst�tzung danken. Dank geb�hrt auch Prof. Dr. Ulrich Schollwoeck und Dr. Fabian Heidrich-Meisner welche mir die M�glichkeit gaben, in einer freundlichen Arbeitsatmosph�re meinen Hobbies, der Physik und der Informatik, nachzugehen!

Vielen Dank und Viel Spa�\\
\vspace{6mm}
\begin{tabularx}{\textwidth}{@{}X@{}@{}r@{}}
Lukas B. Lentner&M�nchen, \date\\
\email{kontakt@lukaslentner.de}
\end{tabularx}