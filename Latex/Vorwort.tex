\newpage
\thispagestyle{empty}
\cleardoublepage
\chapter*{Vorwort}
\addcontentsline{toc}{chapter}{Vorwort}

\begingroup
\leftskip=10mm
\textit{�Man versteht etwas nicht wirklich,\\
wenn man nicht versucht, es zu implementieren.�}\\[2mm]
--- von \textsc{Donald Ervin Knuth} \cite{Knuth}
\vspace{5mm}
\par
\endgroup

TODO: Dank an Prof. Dr. Ulrich Schollw�ck und Dr. Fabian Heidrich-Meisner
TODO: Dank an Mutter, Vater und Bruder
TODO: Dank an Freundin

Generelles TODO:

-----------------------------------------------

Noch machen: Einleitung, Vorwort und Zusammenfassung

F�r Einleitung sind (Etwas Interessantes, Literatur�bersicht und SectionOutline wichtig), Beschreibe andere Methoden: Analytisch Besel, DMRG, Exakt Diagonalisation auch begrenzt, da SSE durch Sign Problem begrentzt, Wei�sche Bezirke und CorrLength erw�hnen einf�hren (Verhalten beim kritischen Punkt, auch grundzustand und sog. Thermisches Chaos)

�berpr�fe Formalit�ten:
\begin{itemize}
	\item Cite im Verzeichnis
	\item F�r jedes Float ordentliche Untershcrift und Verzeichniseintrag
	\item F�r jedes Float die Quelle angeben
	\item Noch andere Literatur wie im Ref und Nolting ...
	\item �berpr�fe: �berschriften, Rechtschreibung, Aufbau, Mehrfacherw�hung
\end{itemize}

Bewertung:
\begin{itemize}
	\item Beschreibe Ergebnisse
	\item Diskussion der Ergebnisse
	\item Stelle heraus was du gemacht hast!
	\item Zeige, dass du es verstanden hast
\end{itemize}


---------------------------------------------

\vspace{6mm}
\begin{tabularx}{\textwidth}{@{}X@{}@{}r@{}}
Lukas B. Lentner&M�nchen, \date\\
\email{kontakt@lukaslentner.de}
\end{tabularx}