\newpage
\thispagestyle{empty}
\cleardoublepage
\chapter{Zusammenfassung}

Die {\bfseries Stochastic Series Expansion} stellt ein m�chtiges Werkzeug f�r das Sampling von Operatorstrings innerhalb eines quantenmechanischen L�sungsansatzes dar. Sie ist f�r gr��ere 2-dimensionale Systeme der schnellste Weg, um die typischen, thermodynamischen Gr��en zu messen und wird hierf�r an mehreren Instituten erfolgreich eingesetzt. Der Algorithmus ist relativ leicht zu implementieren und arbeitet �u�erst speichersparend.

Im Rahmen dieser Arbeit wurde ein Simulationsprogramm geschrieben, welches nicht nur ein blo�es SSE Modul enth�lt, sondern die M�glichkeit anbietet, die SSE-Daten f�r kleine Systeme mit ED {\bfseries Exakt Diagonalisation} zu �berpr�fen, dar�ber hinaus kann ein Zusatzmodul f�r das numerische L�sen des klassischen Ising-Modells eingesetzt werden. Da das Sofware-Projekt objekt-orientiert ausgelegt ist, k�nnen beliebige Komponenten, wie mit einem Baukastensystem zusammengestellt werden. Es ist dadurch sehr flexibel.

Die durchgef�hrten Messungen best�tigten stets die Theorie und liefern besitzen nur einen sehr geringen Fehler. 

Physikalische Sachverhalte werden an mehreren Beispielen/Bildern erkl�hrt und die Abschnitte "`Methode"' in den beiden Projektskapitel 3 und 4 f�hren ausf�hrliche Beschreibungen der 3 Algorithmen an.