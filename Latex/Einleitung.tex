\newpage
\thispagestyle{empty}
\cleardoublepage
\chapter{Einleitung}

\begin{quote}
\itshape Ein Kind sau�t auf Schlittschuhen auf einem zugefrorenen See umher.\\Der Wind pfeift ihm um die Nase ...
\end{quote}

Es ist eine selbstverst�ndliche Naturerfahrung, dass physikalische Stoffe -- wie in diesem Fall das Wasser (H2O) -- in klar definierten Aggregatszust�nden auftreten. Bei genauerem Hinsehen entpuppt sich dieser Sachverhalt jedoch als hochgradig nichttrivial.

Im Zentrum der Entwicklung der modernen Thermodynamik steht die statistische Interpretation von makroskopischen Gr��en wie W�rme, Druck, etc. zuerst als mikroskopische Zitterbewegungen der zugrundeliegenden Molek�le und schlussendlich als rein statistische Effekte eines abstrakten Zustandsraumes, wobei die Temperatur (in Form des Boltzmann-Faktors) eine Abw�gung zwischen h�ufigen und niedrig-energetischen Zust�nden trifft. W�hrend im Grenzfall niedriger Temperaturen die Energieminimierung �berwiegt und letztendlich einen {\bfseries Grunzustand} auspr�gt dominiert f�r hohe Temperaturen die statistische Gleichverteilung {\bfseries (thermisches Chaos)}. Umso mehr muss es verwundern, dass ein derartiges System bei bestimmten, scheinbar willk�rlichen Punkten pl�tzlich das sprunghafte Verhalten eines {\bfseries Phasen�bergangs} zeigt (siehe \cite{Buch}).

Die moderne Sicht auf dieses Ph�nomen geht zur�ck auf Ernst Ising, der 1924 auf Anregung seines Doktorvaters Wilhelm Lenz ein einfaches Modell linear gekoppelter Elementarmagneten untersuchte. Schon l�nger war bekannt, dass sich Ferromagneten oberhalb einer sog. Curie-Temperatur aprupt zu einem Paramagneten wandeln; ein Effekt analog zu oben beschriebenen Phasen�berg�ngen, den Ising auf rein statistischem Wege (mit den Methoden der Thermodynamik) zu deuten suchte. Seine mathematische Auswertung ergab jedoch keinen Phasen�bergang. Erst im Jahre ??? gelang Onsager?? der streng analytische Nachweis eines solchen Phasen�bergangs im 2-dimensionalen (Gitter-) Ising-Modell.

Seither ist dieses Modell eines der am h�ufigsten untersuchten der statistischen Physik, und �hnliche Ph�nomene wurden alsbald in vielerlei anderen Kontexten bekannt; in jedem der F�lle zeigt ein lokal schwach gekoppeltes System, wenn es im {\bfseries thermodynamischen Limes} sehr vieler Teilchen betrachtet wird, bei bestimmten {\bfseries kritischen Temperaturen} diskontinuirliches Verhalten: Die {\bfseries Korrelationsl�nge} $\xi$, welche die Reichweite der mittelbaren Wechselwirkung beschreibt, divergiert an dieser Stelle und es kommt zur charakteristischen Auspr�gung makroskopischer Cluster, sogenannter {\bfseries Wei�scher Bezirke}, w�hrend die beteiligten Ordnungsparameter ein bemerkenswert universelles Verhalten zeigen.

Die Erforschung dererlei Effekte zieht sich heutzutage von verschiedensten Gebieten der Physik �ber rein geometrische (Perkolation) bis hin zu kritischem Verhalten etwa in M�rkten, neuronalen-, und Gennetzwerken. Die wenigsten Anwendungen erlauben dabei allerdings eine geschlosses, analytische L�sung. Vielmehr haben sich derweil Techniken herauskristalisiert, diese (�hnlich gearteten) Probleme zuverl�ssig und schnell numerisch zu behandeln. Dies gilt im besonderen f�r Fragestellungen aus dem Bereich der Quantenmechanik (z.B. Spingl�ser, Gittereichtheorien), da die dort auftretenden, typischen Operatoren (anstelle normaler Zahlen) f�r zus�tzliche Komplexit�t sorgt.

Eine sehr erfolgreicher Herangehensweise besteht hierbei in der von D.C Handscomb \cite{Handscomb} 1962 vorgestellten Reihenentwicklung, der (operatorwertigen) Boltzmann-Faktoren und anschlie�ender kombinatorischer Behandlung aller auftretender Operatorprodukte {\bfseries(Operatorstring)}. Um 1991 gelang es Anders W. Sandvik, Olaf F. Sylju\aa sen und Juhani Kurkij�rvi (siehe \cite{Sandvik}), diesen Ansatz zu einem praktikablen, numerischen Algorithmus weiterzuentwickeln, indem sie ihn mit der klassischen Monte-Carlo Methode zur stochastischen Auswertung a priori unbekannter Funktionen kombinierten {\bfseries(Stochastic Series Expansion)} und einige wesentliche Neuerungen wie das Loop Update, welches eine systematische Analyse der Operatorstrings vollzieht (siehe z.B. \cite{Diplom}).

Ziel der vorliegenden Bachelorarbeit war es zuersteinmal sich in das Themengebiet der Monte-Carlo Methode einzuarbeiten (Kapitel 2) und eine Verbindung zur Vorlesung der Statistischen Physik herzustellen, welche der Autor im Winter geh�rt hatte. Dabei wurde nebenbei ein Simulationsprogramm f�r das 2-dimensionale Ising-Modell erstellt (Kapitel 3.  